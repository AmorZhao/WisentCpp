%------------------------------------------------------------------------------
% Reflections Chapter: IET Competencies (Instructions and Guidance)
%
% This Chapter is **mandatory for passing your project**, although it does
% **not contribute to your project mark**. Its purpose is to demonstrate your
% understanding and application of the **IET engineering competencies** as they
% relate specifically to your project work.
%
% To pass, you must:
% - Clearly address **each required competency** in relation to your project.
% - Include this Chapter in your **draft report** submitted to your supervisor.
% - Obtain **specific feedback** confirming that each competency is adequately covered.
%
% If any part of this Chapter is found to be inadequate, you will be required to
% revise it as directed by your supervisor. **Failure to adequately address any
% of the required competencies as directed may result in failing the project**,
% regardless of other performance.
%
% _While it is expected that you have developed these competencies during your
% degree or prior work experience, clearly articulating them here is essential._
%
% -----------------------------------------------------------------------------
% Requirements:
%
% - **MEng students:** Must complete and pass **two** required competency sections.
%
% The sections below are expected to be achievable for all students. While they
% must meet a minimum standard of adequacy, they are not intended to be overly
% challenging. To help you meet this standard, your supervisor will provide
% feedback on your draft report, highlighting any sections that may be inadequate.
% You will then have the opportunity to revise and improve these areas.
%
% There is considerable flexibility in how you approach these sections, and a
% wide range of content will be considered acceptable. Although you will not
% receive formal instruction on how to complete them specifically in relation to
% your Final Year Project (FYP), you are encouraged to read the guidance notes
% provided, conduct your own web research (for example by using LLMs:
% https://imperiallondon.sharepoint.com/:u:/r/sites/EEEng/FYP/SitePages/Large-language-Model-and-Other-AI-Tools.aspx?csf=1&web=1&e=BJLrF4),
% reflect on your personal experience as an engineer, and be creative in your
% responses. Each section includes an exemplary list of potential topics — you
% are not required to address all of them, and may include other topics.
%
% It is generally expected that you will write around 200 words per section.
% The total word count should typically not exceed 500 words for MEng students
% or 1500 words for BEng students, though longer submissions will be accepted.
%
% Please note that this chapter does not contribute to the overall grade of your
% project. Instead, it is assessed on a pass/fail basis for each section, and
% you must pass all sections to pass the project.
%
% -----------------------------------------------------------------------------

\section*{Introduction}

Throughout my final year project, I have developed and demonstrated several key engineering competencies as outlined by the Institution of Engineering and Technology (IET). In this reflection, I will focus on the areas of effective communication and evaluation of environmental and societal impact, and address how I have met them through my project work. 

% (54 words)

% Communication
% -------------
% The student can communicate effectively on complex engineering matters with technical and non-technical audiences, evaluating the effectiveness of the methods used. 
%
% You should, in this section, discuss briefly the relative merits of your Final Report and your Presentation or Poster as ways to communicate your project content.  
% The deliverables themselves will show that you can communicate effectively to technical and non-technical audiences. This section must show that you can evaluate the use of these different methods.

\section{Communication}

Engineering projects often involve high technical complexity and require clear communication to ensure that all stakeholders understand the objectives, methodologies, and outcomes. In my project, I practised to effectively communicate my work to both technical and non-technical audiences through various mediums, including a technical report, a presentation and informal discussions within the research team. 

The final report serves as the primary medium for documenting detailed technical aspects of the project. It enables me to present a thorough review of state-of-the-art solutions, justify my design decisions and methodologies, and include detailed evaluation results supported by figures, analysis, and references. This format is particularly effective for readers with a technical background who require in-depth understanding, and is essential for long-term reference. 

In contrast, the project presentation is designed for concise and accessible communication, targeting a broader audience, including those without deep expertise in the field. To clearly convey the core concepts, I plan to include visual elements such as diagrams, charts, and concise bullet points in my presentation slides, along with a live demonstration of the code to highlight its usability in practice. The presentation format also offers the advantage of real-time interaction, which will give me the flexibility to present my explanations based on the audience's level of understanding. 

During the design and implementation stages, I actively collaborated with members of my supervisor's research team. This involved discussing design decisions, providing progress updates, and seeking feedback on technical challenges. I used various communication channels such as email, Slack messages and in-person meetings, and spent considerable time studying the team's legacy code and academic papers. These communication sessions are both flexible and highly effective for getting timely feedback, aligning my work with the team's objectives, and staying informed about current research developments.

% The above practices have enhanced my communication skills and helped me to learn to balance technical depth with clarity, ensuring that my work is comprehensible and engaging for both technical and non-technical audiences.

% (290 words)


% Environmental & Societal Impact
% -------------------------------
% The student can evaluate the environmental and societal impact of solutions to complex problems (to include the entire life-cycle of a product or process) and minimise adverse impacts.
%
% You may consider here either the use of your work as a product in an example application, or the process of doing the project. We do not require you to have done your project with the absolute minimum impact, but you should reflect on what are the impacts and what could minimise them. For example: electricity, paper, hardware, etc.
%
% The requirement that you consider the entire life-cycle means, to take two examples, that disposal of old equipment and product carbon footprint are relevant.
%------------------------------------------------------------------------------

\section{Environmental and Societal Impact}

While this project itself does not involve physical products or processes with significant environmental footprints, I took steps to reduce unnecessary resource use and maintained awareness of the ethical and societal dimensions of my work, particularly in the context of software development and data processing.

Throughout the development and evaluation phases, the most significant resource consumption was related to electricity usage during code executions. Minimising it requires writing efficient code and optimising algorithms to reduce unnecessary computation. 

The core focus of my project was on optimising efficiency and system performance for large-scale data processing. If successfully adopted in real-world applications, the resource-conscious solution of handling large datasets could reduce energy consumption associated with data-intensive operations, which is becoming a growing concern in modern computing due to the increasing scale of data and its associated environmental footprint.

In terms of societal impact, the project was conducted within an academic research context. As such, the immediate societal influence is limited. However, the project has potential in long-term applications in fields like large-scale database systems and data transport frameworks. Improved efficiency in these areas could support more scalable and accessible information systems, particularly in data-heavy industries. Ethical concerns such as data privacy, system misuse, or unintended bias could arise. It is therefore important to have responsible policies and implement safeguards such as encryption or access control in data processing to mitigate these risks. 

% (230 words)
